\chapter{Notazione scientifica}
\section{Introduzione}
Consideriamo qualche quantità fisica per esempio massa del protone \SI{0,00000000000000000000000000167262171}{\kilogram}


\begin{table}
	\centering
	\begin{tabular}{ls}
		\toprule
		Nome&Valore\\
		\midrule
	Massa protone&\SI{0,00000000000000000000000000167262171}{\kilogram}\\
	Massa elettrone&\SI{0,00000000000000000000000000000091093826}{\kilogram}\\
	Massa sole&\SI{19888920000000000000000000000000}{\kilogram}\\
		\bottomrule
	\end{tabular}
\caption{Costanti fisiche}
\label{tab:costantifisiche1}
\end{table}
\label{cha:NotazioneScientifica}
\begin{definizionet}{Notazione scientifica}{}\index{Notazione!scientifica}
Un numero è scritto in notazione scientifica se è della forma:
\[X,YYYY \cdot 10^n \]
\begin{tabular}{ll}
dove:&    $X=1,2,3,4,5,6,7,8,9$  \\ 
	    &     $ Y=0,1,2,3,4,5,6,7,8,9$ \\ 
	   &  $n=\pm 1, \pm 2,\pm 3,\ldots$\\
\end{tabular}
 
il numero delle cifre $YYYY$ dipende dal grado di precisione desiderato
\end{definizionet}
\section{Convertire un numero in Notazione scientifica}
\begin{esempiot}{}{}
	$1,412\cdot 10^3$ è in notazione scientifica
	
	$14,12\cdot 10^2$ non è in notazione scientifica
\end{esempiot}