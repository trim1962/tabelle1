\chapter{Monomi}
\label{cha:monomi}
\section{Definizioni}
\begin{definizionet}{Monomio}{}
	Un monomio è il prodotto fra numeri e lettere
\end{definizionet}\index{Monomio}
\begin{esempiot}{Monomio}{}
I seguenti esempi sono tutti monomi
\end{esempiot}
\begin{align*}
	m\cdot a\cdot m\cdot m\cdot a&\\
	a^2\cdot m^3&\\
	2\cdot a\cdot b\cdot3\cdot c\cdot a&\\
	2\cdot x\cdot x&\\
		2\cdot x^2&\\
		\frac{2}{3}\cdot x&\\
\end{align*}
\begin{esempiot}{Non monomio}{}
Quello che segue  non sono monomi
\end{esempiot}
\begin{align*}
a+b&\\
\frac{a}{d+c}&\\
x\div(x+y)&\\
		2\cdot x^{-2}&\\
\end{align*}
\begin{definizionet}{Forma normale monomio}{}
Un monomio è  in forma normale se è formato da un solo numero che chiameremo parte numerica e dal prodotto di potenze con basi letterali che compaiono una sola volta, la parte letterale.
\end{definizionet}\index{Monomio!Forma!normale}\index{Monomio!parte!numerica}\index{Monomio!parte!letterale}
\begin{esempiot}{Forma normale}{}
	I seguenti monomi sono in forma normale
\end{esempiot}
\begin{align*}
	2x^3y&\\
	a^2m^3&\\
	\frac{5}{4}\cdot x^2y&\\
	2&\\
\end{align*}
\begin{table}\centering
	\begin{tabular}{*{3}{>{$}c<{$}}}
		\toprule
		\multicolumn{1}{c}{Monomio}	& \multicolumn{1}{c}{Partte numerica} &\multicolumn{1}{c}{Parte letterale}  \\
		\midrule
	3a	& +3 & a \\
	-5x^2y^3	&-5  & x^2y^3 \\
	+\frac{2}{5}a^2b^3z^4&+\frac{2}{5}&a^2b^3z^4\\
	\frac{1}{2}mv^2&\frac{1}{2}&mv^2\\
		\bottomrule
	\end{tabular}
	\caption{Parte Numerica e letterale}
\end{table}
\begin{definizionet}{Monomio nullo}{}
	Un monomio è nullo se ha parte numerica zero.
\end{definizionet}\index{Monomio!zero}\index{Monomio!nullo}
\begin{esempiot}{Monomio zero}{}
\begin{align*}
	0x^4y^2&\\
	0b^2c^3&\\
	0x^3yz&\\
	0&\\
\end{align*}
\end{esempiot}
\begin{definizionet}{Monomi simili}{}
	Due monomi sono simili se hanno la stessa parte letterale
\end{definizionet}\index{Monomio!simile}
Il concetto di similitudine è usato normalmente nella vita quotidiana. Una penna è simile a un'altra penna. Una matita non è simile a un pennarello, una mela è simile ad una mela ma non è simile ad un peperone. Il nostro cervello naturalmente classifica gli oggetti che ci circondano e questo ci permette d'interagire   con loro. Se cerco della frutta dolce prendo un mandarino e non un limone, infatti un limone non è simile a un mandarino.  
\begin{esempiot}{Monomi simili}{}
	I seguenti monomi sono simili a coppie.
\end{esempiot}
\begin{align*}
	4a&&5a&\\
	-3xy^2&&\frac{2}{3}xy^2&\\
	x&&5x&\\
	3z&&z5\\
\end{align*}
\begin{definizionet}{Monomi opposti}{}
	Due monomi  simili sono opposti se hanno la parte numerica opposta.
\end{definizionet}\index{Monomio!opposto}
Quindi per definire un monomio opposto devo porre due condizioni.
\begin{esempiot}{Monomi opposti}{}
	I seguenti monomi sono opposti a coppie
\end{esempiot}
\begin{align*}
	4a&&-4a&\\
	-3xy^2&&3xy^2&\\
	-x&&x&\\
	-5z&&z5\\
\end{align*}
\begin{definizionet}{Monomi opposti}{}
	Due monomi  simili sono uguali se hanno la parte numerica uguale.
\end{definizionet}\index{Monomio!uguale}
Anche in questo caso dobbiamo verificare due condizioni.
\begin{definizionet}{Grado rispetto ad una lettera}{}
Il grado di un monomio rispetto a una lettera è l'esponente con cui appare questa lettera. Se una lettera non compare il suo grado è zero.
\end{definizionet}
\begin{esempiot}{Grado rispetto alla lettera}{}
	Calcolare il grado rispetto alla lettera del monomio
\end{esempiot}
\[3a^3b^2c^4d\] Per la $a$, questo monomio è di terzo grado , di secondo per la $b$, quarto per $c$, primo per $d$, zero per $e$
\section{Operazioni}
\subsection{Somma}
\begin{definizionet}{Somma}{}
La somma fra monomi simili, è un monomio che ha per somma la somma algebrica delle parti numeriche e per parte letterale la stessa parte letterale.  
\end{definizionet}
La somma di due oggetti simili è un oggetto simile. Aggiungendo tre cacciaviti ad altri cinque  otteniamo otto cacciativi non otto martelli.