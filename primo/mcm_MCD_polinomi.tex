\chapter{MCD e mcm}
\label{cha:mcmmcdpolinomi}
\section{MCD}\index{Polinomi!MCD}
\label{secMCDpolinomi}
Il $\mcd$ fra più polinomi si ottiene moltiplicando i fattori comuni presi una sola volta, con il minore esponente.
\begin{esempiot}{}{}
Trovare il $\mcd$ fra $a^3-2a^2b$ e $a^2-4b^2$
\end{esempiot}
 i due polinomi sono somme di addendi quindi 
\begin{center}
\begin{tikzpicture}
\tikzset{notondo/.style={ellipse},
	tondo/.style={draw,notondo},
	linea/.style={very thick},
	freccia/.style={-triangle 90,linea},
	etichetta/.style={tondo,node distance = 2cm,align=center},
	noetichetta/.style={},
	freccialta/.style={freccia,bend left=45},
	frecciabassa/.style={freccia,bend right=45}};
\node [notondo] (p1)  {$a^3$};
\node [notondo,right of=p1] (p2)  {$-$};
\node [notondo,right of=p2] (p3)  {$2a^2b$};
\node [notondo,right of=p3] (p4)  {};
\node [notondo,right of=p4] (p5)  {$a^2$};
\node [notondo,right of=p5 ] (p6)  {$-$};
\node [notondo,right of=p6] (p7)  {$4b^2$};
\node [etichetta,below of=p6] (p9)  {Addendi};
\node [etichetta,below of=p2] (p8)  {Addendi};
\draw[freccia] (p8)--(p1);
\draw[freccia] (p8)--(p3);
\draw[freccia] (p9)--(p5);
\draw[freccia] (p9)--(p7);
\end{tikzpicture}
\end{center}
raccogliendo a fattore comune nel primo e osservando che nel secondo abbiamo una differenza di quadrati otteniamo 
\begin{center}
\begin{tikzpicture}
\tikzset{notondo/.style={ellipse},
	tondo/.style={draw,notondo},
	linea/.style={very thick},
	freccia/.style={-triangle 90,linea},
	etichetta/.style={tondo,node distance = 2cm,align=center},
	noetichetta/.style={},
	freccialta/.style={freccia,bend left=45},
	frecciabassa/.style={freccia,bend right=45}};
\node [notondo,node distance=0.5cm] (p1)  {$a^2$};
\node [notondo,right of=p1] (p2)  {};
\node [notondo,right of=p2,,node distance=0.1cm] (p3)  {$(a-2b)$};
\node [notondo,right of=p3,node distance=2cm ] (p4)  {};
\node [notondo,right of=p4,] (p5)  {$(a-2b)$};
\node [notondo,right of=p5,,node distance=0.5cm ] (p6)  {};
\node [notondo,right of=p6] (p7)  {$(a+2b)$};
\node [etichetta,below of=p4] (p9)  {Fattori\\ comuni};
\node [etichetta,above of=p4] (p8)  {Fattori\\ non comuni};
\draw[freccia] (p8)--(p1);
\draw[freccia] (p8)--(p7);
\draw[freccia] (p9)--(p3);
\draw[freccia] (p9)--(p5);
\end{tikzpicture}
\end{center}
Abbiamo trasformato le due somme in prodotti di fattori tramite un raccoglimento totale. Vi è un solo fattore comune, il $\mcd$ fra i due polinomi sarà\[(a-2b)\]
\begin{esempiot}{}{}
Trovare il $\mcd$ fra $x^2+9-6x$ e $x^2-2x-3$
\end{esempiot}
 i due polinomi sono somme di addendi quindi:
\begin{center}
\begin{tikzpicture}
\tikzset{notondo/.style={ellipse},
	tondo/.style={draw,notondo},
	linea/.style={very thick},
	freccia/.style={-triangle 90,linea},
	etichetta/.style={tondo,node distance = 2cm,align=center},
	noetichetta/.style={},
	freccialta/.style={freccia,bend left=45},
	frecciabassa/.style={freccia,bend right=45}};
\node [notondo] (p1)  {$x^2$};
\node [notondo,right of=p1] (p2)  {$+$};
\node [notondo,right of=p2] (p3)  {$9$};
\node [notondo,right of=p3] (p3a)  {$-$};
\node [notondo,right of=p3a] (p3b)  {$6x$};
\node [notondo,right of=p3b] (p4)  {};
\node [notondo,right of=p4] (p5)  {$x^2$};
\node [notondo,right of=p5 ] (p6)  {$-$};
\node [notondo,right of=p6] (p7)  {$2x$};
\node [notondo,right of=p7] (p7a)  {$-$};
\node [notondo,right of=p7a] (p7b)  {$3$};
\node [etichetta,below of=p7] (p9)  {Addendi};
\node [etichetta,below of=p3] (p8)  {Addendi};
\draw[freccia] (p8)--(p1);
\draw[freccia] (p8)--(p3);
\draw[freccia] (p8)--(p3b);
\draw[freccia] (p9)--(p5);
\draw[freccia] (p9)--(p7);
\draw[freccia] (p9)--(p7b);
\end{tikzpicture}
\end{center}
Il primo è il quadrato di un binomio, l'altro trinomio è un trinomio particolare o somma prodotto. Otteniamo: 
\begin{center}
\begin{tikzpicture}
\tikzset{notondo/.style={ellipse},
	tondo/.style={draw,notondo},
	linea/.style={very thick},
	freccia/.style={-triangle 90,linea},
	etichetta/.style={tondo,node distance = 2cm,align=center},
	noetichetta/.style={},
	freccialta/.style={freccia,bend left=45},
	frecciabassa/.style={freccia,bend right=45}};
%\node [notondo,node distance=0.5cm] (p1)  {$(x-3)^2$};
%\node [notondo,right of=p1] (p2)  {$$};
\node [notondo] (p3)  {$(x-3)^2$};
\node [notondo,right of=p3,node distance=2cm ] (p4)  {};
\node [tondo,right of=p4,] (p5)  {$(x-3)$};
\node [notondo,right of=p5,,node distance=0.5cm ] (p6)  {$ $};
\node [notondo,right of=p6] (p7)  {$(x+1)$};
\node [etichetta,below of=p4] (p9)  {Fattori\\ comuni};
\node [etichetta,above of=p7] (p8)  {Fattore \\non comune};
\node [etichetta,above of=p3,] (p10)  {Minore \\ esponente};
%\draw[freccia] (p8)--(p1);
\draw[freccia] (p8)--(p7);
\draw[freccia] (p9)--(p3);
\draw[freccia] (p9)--(p5);
\draw[freccia] (p10)--(p4);
\end{tikzpicture}
\end{center}
Abbiamo trasformato le due somme in  prodotti di fattori. I fattori in questo caso sono due. Viene preso il fattore comune con il minore esponente. Il $\mcd$ fra i due polinomi sarà\[x-3 \]

\begin{esempiot}{}{}
Trovare il $\mcm$ fra $x^3-2x^2$, $x^2-4x+4$ e $ x^3-4x$
\end{esempiot}
 i tre polinomi sono somme di addendi quindi 
\begin{center}
\begin{tikzpicture}
\tikzset{notondo/.style={ellipse},
	tondo/.style={draw,notondo},
	linea/.style={very thick},
	freccia/.style={-triangle 90,linea},
	etichetta/.style={tondo,node distance = 2cm,align=center},
	noetichetta/.style={},
	freccialta/.style={freccia,bend left=45},
	frecciabassa/.style={freccia,bend right=45}};
\node [notondo] (p1)  {$x^2$};
\node [notondo,right of=p1] (p2)  {$-$};
\node [notondo,right of=p2] (p3)  {$2x$};
\node [notondo,right of=p3] (p4)  {};
\node [notondo,right of=p4] (p5)  {$x^2$};
\node [notondo,right of=p5 ] (p6)  {$-$};
\node [notondo,right of=p6] (p7)  {$4x$};
\node [notondo,right of=p7] (p7a)  {$+$};
\node [notondo,right of=p7a] (p7b)  {$4$};
\node [notondo,right of=p7b] (a1)  {};
\node [notondo,right of=a1] (a2)  {$x^3$};
\node [notondo,right of=a2] (a3)  {$-$};
\node [notondo,right of=a3] (a4)  {$4x$};

\node [etichetta,below of=p2] (p8)  {Addendi};
\node [etichetta,below of=p7] (p9)  {Addendi};
\node [etichetta,below of=a3] (p10)  {Addendi};
\draw[freccia] (p8)--(p1);
\draw[freccia] (p8)--(p3);
\draw[freccia] (p9)--(p5);
\draw[freccia] (p9)--(p7);
\draw[freccia] (p9)--(p7b);
\draw[freccia] (p10)--(a2);
\draw[freccia] (p10)--(a4);
\end{tikzpicture}
\end{center}
Il primo lo fattorizzo con un raccoglimento totale, ho poi un quadrato, l'altro binomio un altro raccoglimento totale e una differenza di quadrati. Otteniamo: 
\begin{center}
\begin{tikzpicture}
\tikzset{notondo/.style={ellipse},
	tondo/.style={draw,notondo},
	linea/.style={very thick},
	freccia/.style={-triangle 90,linea},
	etichetta/.style={tondo,node distance = 2cm,align=center},
	noetichetta/.style={},
	freccialta/.style={freccia,bend left=45},
	frecciabassa/.style={freccia,bend right=45}};
\node [notondo] (p1)  {$x^2$};
\node [notondo,right of=p1] (p2)  {};
\node [tondo,right of=p2,node distance=0.5cm] (p3)  {$(x-2)$};
\node [notondo,right of=p3,node distance=2cm ] (p4)  {};
\node [notondo,right of=p4,] (p5)  {$(x-2)^2$};
\node [notondo,right of=p5,node distance=2cm ] (p6)  {};
\node [notondo,right of=p6] (p7)  {$x$};
\node [notondo,right of=p7] (a1)  {};
\node [notondo,right of=a1,node distance=0.cm] (a2)  {$(x-2)$};
\node [notondo,right of=a2] (a3)  {};
\node [notondo,right of=a3,node distance=0.4cm] (a4)  {$(x+2)$};
\node [etichetta,below of=p5] (p9)  {Fattori\\ comuni};
\node [etichetta,above of=p4] (p8)  {Fattori \\non comuni};
\node [etichetta,below of=p3,] (p10)  {Minore \\ esponente};
\draw[freccia] (p8)--(p7);
\draw[freccia] (p8)--(a4);
\draw[freccia] (p8)--(p1);
\draw[freccia] (p9)--(p3);
\draw[freccia] (p9)--(p5);
\draw[freccia] (p9)--(a2);
\draw[freccia] (p10)--(p3);
%\draw[freccia] (p10)--(p5);
\end{tikzpicture}
\end{center}
Abbiamo trasformato le tre somme in  prodotti di fattori. Vi è un solo fattore comune ed è preso quello con il minore esponente. Il $\mcd$ fra i tre polinomi sarà\[x-2 \]
\section{mcm}\index{Polinomi!mcm}
\label{sec:mcmpolinomi}
Il $\mcm$ fra più polinomi si ottiene moltiplicando i fattori comuni  e non comuni, presi una sola volta, con il maggiore esponente.

In genere, i polinomi sono una somma di più termini (addendi) quindi non è possibile,in genere, calcolare subito il $\mcm$.  
\begin{esempiot}{}{}
Trovare il $\mcm$ fra $a^3-2a^2b$ e $a^2-4b^2$ 
\end{esempiot}
i due polinomi sono somme di addendi quindi 
\begin{center}
\begin{tikzpicture}
\tikzset{notondo/.style={ellipse},
	tondo/.style={draw,notondo},
	linea/.style={very thick},
	freccia/.style={-triangle 90,linea},
	etichetta/.style={tondo,node distance = 2cm,align=center},
	noetichetta/.style={},
	freccialta/.style={freccia,bend left=45},
	frecciabassa/.style={freccia,bend right=45}};
\node [notondo] (p1)  {$a^3$};
\node [notondo,right of=p1] (p2)  {$-$};
\node [notondo,right of=p2] (p3)  {$2a^2b$};
\node [notondo,right of=p3] (p4)  {};
\node [notondo,right of=p4] (p5)  {$a^2$};
\node [notondo,right of=p5 ] (p6)  {$-$};
\node [notondo,right of=p6] (p7)  {$4b^2$};
\node [etichetta,below of=p6] (p9)  {Addendi};
\node [etichetta,below of=p2] (p8)  {Addendi};
\draw[freccia] (p8)--(p1);
\draw[freccia] (p8)--(p3);
\draw[freccia] (p9)--(p5);
\draw[freccia] (p9)--(p7);
\end{tikzpicture}
\end{center}
raccogliendo a fattore comune nel primo e osservando che nel secondo abbiamo una differenza di quadrati otteniamo 
\begin{center}
\begin{tikzpicture}
\tikzset{notondo/.style={ellipse},
	tondo/.style={draw,notondo},
	linea/.style={very thick},
	freccia/.style={-triangle 90,linea},
	etichetta/.style={tondo,node distance = 2cm,align=center},
	noetichetta/.style={},
	freccialta/.style={freccia,bend left=45},
	frecciabassa/.style={freccia,bend right=45}};
\node [notondo,node distance=0.5cm] (p1)  {$a^2$};
\node [notondo,right of=p1] (p2)  {};
\node [notondo,right of=p2,,node distance=0.1cm] (p3)  {$(a-2b)$};
\node [notondo,right of=p3,node distance=2cm ] (p4)  {};
\node [notondo,right of=p4,] (p5)  {$(a-2b)$};
\node [notondo,right of=p5,,node distance=0.5cm ] (p6)  {};
\node [notondo,right of=p6] (p7)  {$(a+2b)$};
\node [etichetta,below of=p4] (p9)  {Fattori\\ comuni};
\node [etichetta,above of=p4] (p8)  {Fattori\\ non comuni};
\draw[freccia] (p8)--(p1);
\draw[freccia] (p8)--(p7);
\draw[freccia] (p9)--(p3);
\draw[freccia] (p9)--(p5);
\end{tikzpicture}
\end{center}
Abbiamo trasformato le due somme in un prodotti di fattori. Possiamo dividere i fattori  in fattori comuni e in fattori non comuni. Il $\mcm$ fra i due polinomi sarà\[a^2(a-2b)(a+2b) \]
\begin{esempiot}{}{}
Trovare il $\mcm$ fra $x^2+9-6x$ e $x^2-2x-3$ 
\end{esempiot}
i due polinomi sono somme di addendi quindi 
\begin{center}
\begin{tikzpicture}
\tikzset{notondo/.style={ellipse},
	tondo/.style={draw,notondo},
	linea/.style={very thick},
	freccia/.style={-triangle 90,linea},
	etichetta/.style={tondo,node distance = 2cm,align=center},
	noetichetta/.style={},
	freccialta/.style={freccia,bend left=45},
	frecciabassa/.style={freccia,bend right=45}};
\node [notondo] (p1)  {$x^2$};
\node [notondo,right of=p1] (p2)  {$+$};
\node [notondo,right of=p2] (p3)  {$9$};
\node [notondo,right of=p3] (p3a)  {$-$};
\node [notondo,right of=p3a] (p3b)  {$6x$};
\node [notondo,right of=p3b] (p4)  {};
\node [notondo,right of=p4] (p5)  {$x^2$};
\node [notondo,right of=p5 ] (p6)  {$-$};
\node [notondo,right of=p6] (p7)  {$2x$};
\node [notondo,right of=p7] (p7a)  {$-$};
\node [notondo,right of=p7a] (p7b)  {$3$};
\node [etichetta,below of=p7] (p9)  {Addendi};
\node [etichetta,below of=p3] (p8)  {Addendi};
\draw[freccia] (p8)--(p1);
\draw[freccia] (p8)--(p3);
\draw[freccia] (p8)--(p3b);
\draw[freccia] (p9)--(p5);
\draw[freccia] (p9)--(p7);
\draw[freccia] (p9)--(p7b);
\end{tikzpicture}
\end{center}
Il primo è il quadrato di un binomio, l'altro trinomio è un trinomio particolare o somma prodotto. Otteniamo: 
\begin{center}
\begin{tikzpicture}
\tikzset{notondo/.style={ellipse},
	tondo/.style={draw,notondo},
	linea/.style={very thick},
	freccia/.style={-triangle 90,linea},
	etichetta/.style={tondo,node distance = 2cm,align=center},
	noetichetta/.style={},
	freccialta/.style={freccia,bend left=45},
	frecciabassa/.style={freccia,bend right=45}};
%\node [notondo,node distance=0.5cm] (p1)  {$(x-3)^2$};
%\node [notondo,right of=p1] (p2)  {};
\node [tondo] (p3)  {$(x-3)^2$};
\node [notondo,right of=p3,node distance=2cm ] (p4)  {};
\node [notondo,right of=p4,] (p5)  {$(x-3)$};
\node [notondo,right of=p5,,node distance=0.5cm ] (p6)  {};
\node [notondo,right of=p6] (p7)  {$(x+1)$};
\node [etichetta,below of=p4] (p9)  {Fattori\\ comuni};
\node [etichetta,above of=p7] (p8)  {Fattore \\non comune};
\node [etichetta,above of=p3,] (p10)  {Maggiore \\ esponente};
%\draw[freccia] (p8)--(p1);
\draw[freccia] (p8)--(p7);
\draw[freccia] (p9)--(p3);
\draw[freccia] (p9)--(p5);
\draw[freccia] (p10)--(p3);
\end{tikzpicture}
\end{center}
Abbiamo trasformato le due somme in prodotti di fattori. I fattori in questo caso sono due. Per i fattori comuni viene preso quello con il maggiore esponente. Il $\mcm$ fra i due polinomi sarà\[(x-3)^2(x+1) \]
\begin{esempiot}{}{}
 Trovare il $\mcm$ fra $x^3-2x^2$, $x^2-4x+4$ e $ x^3-4x$ 
 \end{esempiot}
 i tre polinomi sono somme di addendi quindi 
\begin{center}
\begin{tikzpicture}
\tikzset{notondo/.style={ellipse},
	tondo/.style={draw,notondo},
	linea/.style={very thick},
	freccia/.style={-triangle 90,linea},
	etichetta/.style={tondo,node distance = 2cm,align=center},
	noetichetta/.style={},
	freccialta/.style={freccia,bend left=45},
	frecciabassa/.style={freccia,bend right=45}};
\node [notondo] (p1)  {$x^2$};
\node [notondo,right of=p1] (p2)  {$-$};
\node [notondo,right of=p2] (p3)  {$2x$};
\node [notondo,right of=p3] (p4)  {};
\node [notondo,right of=p4] (p5)  {$x^2$};
\node [notondo,right of=p5 ] (p6)  {$-$};
\node [notondo,right of=p6] (p7)  {$4x$};
\node [notondo,right of=p7] (p7a)  {$+$};
\node [notondo,right of=p7a] (p7b)  {$4$};
\node [notondo,right of=p7b] (a1)  {};
\node [notondo,right of=a1] (a2)  {$x^3$};
\node [notondo,right of=a2] (a3)  {$-$};
\node [notondo,right of=a3] (a4)  {$4x$};

\node [etichetta,below of=p2] (p8)  {Addendi};
\node [etichetta,below of=p7] (p9)  {Addendi};
\node [etichetta,below of=a3] (p10)  {Addendi};
\draw[freccia] (p8)--(p1);
\draw[freccia] (p8)--(p3);
\draw[freccia] (p9)--(p5);
\draw[freccia] (p9)--(p7);
\draw[freccia] (p9)--(p7b);
\draw[freccia] (p10)--(a2);
\draw[freccia] (p10)--(a4);
\end{tikzpicture}
\end{center}
Il primo lo fattorizzo con un raccoglimento totale, il secondo è un quadrato, l'ultimo è un binomio che dopo un  raccoglimento totale da una differenza di quadrati. Otteniamo: 
\begin{center}
\begin{tikzpicture}
\tikzset{notondo/.style={ellipse},
tondo/.style={draw,notondo},
linea/.style={very thick},
freccia/.style={-triangle 90,linea},
etichetta/.style={tondo,node distance = 2cm,align=center},
noetichetta/.style={},
freccialta/.style={freccia,bend left=45},
frecciabassa/.style={freccia,bend right=45}};
\node [tondo] (p1)  {$x^2$};
\node [notondo,right of=p1] (p2)  {};
\node [notondo,right of=p2,node distance=0.5cm] (p3)  {$(x-2)$};
\node [notondo,right of=p3,node distance=2cm ] (p4)  {};
\node [tondo,right of=p4,] (p5)  {$(x-2)^2$};
\node [notondo,right of=p5,node distance=2cm ] (p6)  {};
\node [notondo,right of=p6] (p7)  {$x$};
\node [notondo,right of=p7] (a1)  {};
\node [notondo,right of=a1,node distance=0.cm] (a2)  {$(x-2)$};
\node [notondo,right of=a2] (a3)  {};
\node [notondo,right of=a3,node distance=0.4cm] (a4)  {$(x+2)$};
\node [etichetta,below of=p5] (p9)  {Fattori\\ comuni};
\node [etichetta,above of=p4] (p8)  {Fattori \\non comuni};
\node [etichetta,below of=p3,] (p10)  {Maggiore \\ esponente};

\draw[freccia] (p8)--(p7);
\draw[freccia] (p8)--(a4);
\draw[freccia] (p8)--(p1);
\draw[freccia] (p9)--(p3);
\draw[freccia] (p9)--(p5);
\draw[freccia] (p9)--(a2);
\draw[freccia] (p10)--(p1);
\draw[freccia] (p10)--(p5);
\end{tikzpicture}
\end{center}
Abbiamo trasformato le tre somme in un prodotto di fattori. I fattori in questo caso sono tre. Per i fattori comuni viene preso quello con il maggiore esponente. Il $\mcm$ fra i tre polinomi sarà\[x^2(x-2)^2(x+2) \]

