\chapter{Percentuali}
\section{Parte, tutto e percentuale}
La percentuale\index{Percentuale} è un strumento matematico che rappresenta il rapporto fra due quantità.
\[Percentuale=\dfrac{Parte}{Tutto}\cdot 100\]
Sostanzialmente è una relazione con tre termini, quindi bisogna conoscerne due per ottenerne una terza.
\subsection{Parte e Tutto} 
	\begin{esempiot}{Conosco Parte e tutto}{}
	In una classe vi sono \num{12} maschi e \num{8} femmine. Trovare la percentuale dei ragazzi e delle ragazze.
\end{esempiot}
Calcoliamo la percentuale degli alunni:
\begin{align*}
Tutto=&\num{12}+\num{8}=\num{20}\\
Parte=&\num{12}\\
Percentuale=&\dfrac{Parte}{Tutto}\cdot 100\\
Percentuale=&\dfrac{12}{20}\cdot 100\\
Percentuale=&\SI{60}{\percent}
\end{align*}
Calcoliamo la percentuale delle alunne:
\begin{align*}
	Tutto=&\num{12}+\num{8}=\num{20}\\
	Parte=&\num{8}\\
	Percentuale=&\dfrac{Parte}{Tutto}\cdot 100\\
	Percentuale=&\dfrac{8}{20}\cdot 100\\
	Percentuale=&\SI{40}{\percent}
\end{align*}