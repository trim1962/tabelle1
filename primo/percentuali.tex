\chapter{Percentuali}
\section{Parte, tutto e percentuale}
La percentuale\index{Percentuale} è un strumento matematico che rappresenta il rapporto fra due quantità.
\[Percentuale=\dfrac{Parte}{Tutto}\cdot 100\]
Sostanzialmente è una relazione con tre termini, quindi bisogna conoscerne due per ottenerne una terza.
\subsection{Parte e Tutto} 
	\begin{esempiot}{Conosco Parte e tutto}{}
	In una classe vi sono \num{12} maschi e \num{8} femmine. Trovare la percentuale dei ragazzi e delle ragazze.
\end{esempiot}
Calcoliamo la percentuale degli alunni:
\begin{align*}
Tutto=&\num{12}+\num{8}=\num{20}\\
Parte=&\num{12}\\
Percentuale=&\dfrac{Parte}{Tutto}\cdot 100\\
Percentuale=&\dfrac{12}{20}\cdot 100\\
Percentuale=&\SI{60}{\percent}
\end{align*}
Calcoliamo la percentuale delle alunne:
\begin{align*}
	Tutto=&\num{12}+\num{8}=\num{20}\\
	Parte=&\num{8}\\
	Percentuale=&\dfrac{Parte}{Tutto}\cdot 100\\
	Percentuale=&\dfrac{8}{20}\cdot 100\\
	Percentuale=&\SI{40}{\percent}
\end{align*}
\subsection{Percentuale e Tutto}  
\begin{esempiot}{Conosco Percentuale e Tutto}{}
	In una classe vi sono \num{20} alunni. Trovare quante sono le ragazze sapendo che sono il \SI{40}{\percent}
\end{esempiot}
Troviamo la parte
\begin{align*}
	Tutto=&\num{20}\\
	Percentuale=&\SI{40}{\percent}\\
	Percentuale=&\dfrac{Parte}{Tutto}\cdot 100\\
	40=&\dfrac{Parte}{20}\cdot 100
	\intertext{Isolo $Parte$}
	40\cdot 20=&Parte\cdot 100\\
	Parte=&\dfrac{40\cdot 20}{100}\\
	Parte=&8
\end{align*}
Rileggendo i calcoli possiamo dire che \[Parte=\dfrac{Percentuale\cdot Tutto}{100}\]
\subsection{Percentuale e Parte} 
\begin{esempiot}{Conosco Percentuale e parte}{}
	Dodici ragazzi sono il \SI{60}{\percent} degli alunni di una classe. Da quanti alunni è formata la classe?
\end{esempiot}
Troviamo il Tutto
\begin{align*}
	Parte=&\num{12}\\
	Percentuale=&\SI{60}{\percent}\\
	Percentuale=&\dfrac{Parte}{Tutto}\cdot 100\\
	60=&\dfrac{12}{Tutto}\cdot 100
	\intertext{Isolo $Tutto$}
	Tutto\cdot 60=&12\cdot 100\\
	Tutto=&\dfrac{12}{60}\cdot 100\\
	Tutto=&20
\end{align*}
Rileggendo i calcoli possiamo dire che \[Tutto=\dfrac{Parte}{Percentuale}\cdot 100\]