\chapter{Percentuali}
\section{Parte, tutto e percentuale}
La percentuale\index{Percentuale} è un strumento matematico che rappresenta il rapporto fra due quantità.
\[Percentuale=\dfrac{Parte}{Tutto}\cdot 100\]
Sostanzialmente è una relazione con tre termini, quindi bisogna conoscerne due per ottenerne una terza.
\subsection{Parte e Tutto} 
	\begin{esempiot}{Conosco Parte e tutto}{}
	In una classe vi sono \num{12} maschi e \num{8} femmine. Trovare la percentuale dei ragazzi e delle ragazze.
\end{esempiot}
Calcoliamo la percentuale degli alunni:
\begin{align*}
Tutto=&\num{12}+\num{8}=\num{20}\\
Parte=&\num{12}\\
Percentuale=&\dfrac{Parte}{Tutto}\cdot 100\\
Percentuale=&\dfrac{12}{20}\cdot 100\\
Percentuale=&\SI{60}{\percent}
\end{align*}
Calcoliamo la percentuale delle alunne:
\begin{align*}
	Tutto=&\num{12}+\num{8}=\num{20}\\
	Parte=&\num{8}\\
	Percentuale=&\dfrac{Parte}{Tutto}\cdot 100\\
	Percentuale=&\dfrac{8}{20}\cdot 100\\
	Percentuale=&\SI{40}{\percent}
\end{align*}
\subsection{Percentuale e Tutto}  
\begin{esempiot}{Conosco Percentuale e Tutto}{}
	In una classe vi sono \num{20} alunni. Trovare quante sono le ragazze sapendo che sono il \SI{40}{\percent}
\end{esempiot}
Troviamo la parte
\begin{align*}
	Tutto=&\num{20}\\
	Percentuale=&\SI{40}{\percent}\\
	Percentuale=&\dfrac{Parte}{Tutto}\cdot 100\\
	40=&\dfrac{Parte}{20}\cdot 100
	\intertext{Isolo $Parte$}
	40\cdot 20=&Parte\cdot 100\\
	Parte=&\dfrac{40\cdot 20}{100}\\
	Parte=&8
\end{align*}
Rileggendo i calcoli possiamo dire che \[Parte=\dfrac{Percentuale\cdot Tutto}{100}\]
\subsection{Percentuale e Parte} 
\begin{esempiot}{Conosco Percentuale e parte}{}
	Dodici ragazzi sono il \SI{60}{\percent} degli alunni di una classe. Da quanti alunni è formata la classe?
\end{esempiot}
Troviamo il Tutto
\begin{align*}
	Parte=&\num{12}\\
	Percentuale=&\SI{60}{\percent}\\
	Percentuale=&\dfrac{Parte}{Tutto}\cdot 100\\
	60=&\dfrac{12}{Tutto}\cdot 100
	\intertext{Isolo $Tutto$}
	Tutto\cdot 60=&12\cdot 100\\
	Tutto=&\dfrac{12}{60}\cdot 100\\
	Tutto=&20
\end{align*}
Rileggendo i calcoli possiamo dire che \[Tutto=\dfrac{Parte}{Percentuale}\cdot 100\]
\chapter{Sconto}
\section{Lo Sconto e lo sconto percentuale}
\begin{figure}
	\centering
	\includestandalone{primo/sconto/sconto}
	\caption{Sconto}
	\label{fig:sconto}
\end{figure}
 In un supermercato troviamo sotto un televisore l'etichetta come la~\cref*{fig:sconto}. Abbiamo un prezzo iniziale di \EUR{600}, un prezzo scontato  di \EUR{340}. La differenza tra i due prezzi è lo sconto\index{Sconto}. Vale la seguente relazione:\[ {Prezzo}_{iniziale}- {Prezzo}_{finale}=Sconto\] In questo caso lo sconto è di \EUR{260}. Lo sconto rappresenta il denaro che risparmiamo rispetto al prezzo iniziale.
 
 Possiamo facilmente  definire  uno sconto percentuale. Per definirlo dobbiamo trovare quindi una parte e un tutto. La parte è ovviamente lo $Sconto$. Il tutto è il ${Prezzo}_{iniziale}$ infatti \[{Prezzo}_{iniziale}={Prezzo}_{finale}+Sconto\]Quindi lo $Sconto$ è una parte del ${Prezzo}_{iniziale}$ 
 
 Possiamo definire lo \[{Sconto}_{percentuale}=\dfrac{Sconto}{{Prezzo}_{iniziale}}\cdot 100\]\index{Sconto!percentuale}
 Lo ${Sconto}_{percentuale}$ non è espresso in Euro dato che è un numero puro non un valore monetario come gli altri.
 \subsection{Prezzo iniziale e finale} 
 	\begin{esempiot}{Conosco Prezzo iniziale e quello finale}{}
 	Una giacca con un prezzo iniziale di \EUR{150} viene venduta a \EUR{80}. Calcolare lo sconto percentuale. 
 \end{esempiot}
Per risolvere l'esercizio bisogna prima trovare quanto è l'importo dello sconto e successivamente  la sua percentuale rispetto al costo iniziale. Procediamo come segue:
 \begin{align*}
 	{Prezzo}_{iniziale}=&\mbox{\EUR{150}}\\
 	{Prezzo}_{finale}=&\mbox{\EUR{80}}\\
 	Sconto=&\mbox{\EUR{70}}\\
 	{Sconto}_{percentuale}=&\dfrac{70}{150}\cdot 100\\
 \end{align*}
\subsection{Prezzo iniziale e sconto percentuale}
	\begin{esempiot}{Conosco Prezzo iniziale e sconto percentuale}{}
Un tostapane costa \EUR{40}. Viene scontato del \SI{20}{\percent} quale è il suo prezzo finale?
\end{esempiot}
Per risolvere questo problema bisogna ottenere dallo ${Sconto}_{percentuale}$  lo $Sconto$ e successivamente il ${Prezzo}_{finale}$
\begin{align*}
	{Prezzo}_{iniziale}=&\mbox{\EUR{40}}\\
 	{Sconto}_{Percentuale}=&\SI{20}{\percent}\\
 	Sconto=&\dfrac{{Prezzo}_{iniziale}\cdot {Sconto}_{Percentuale} }{100}\\
 	=&\dfrac{40\cdot 20 }{100}\\
 	=&\mbox{\EUR{8}}\\
 	{Prezzo}_{finale}=&{Prezzo}_{iniziale}-Sconto\\
 	=40-8=&\mbox{\EUR{32}}\\
\end{align*}
\subsection{Prezzo finale e sconto}
	\begin{esempiot}{Conosco Prezzo finale e sconto percentuale}{}
Un libro viene venduto a \EUR{23.75} scontato del \SI{20}{\percent}: Quale era il suo prezzo iniziale?
\end{esempiot}
Per trovare il ${Prezzo}_{iniziale}$ sono necessari alcuni calcoli preliminari
\begin{align*}
	{Prezzo}_{iniziale}=&Sconto+{Prezzo}_{finale}\\
		Sconto=&\dfrac{{Prezzo}_{iniziale}\cdot {Sconto}_{Percentuale} }{100}\\
		{Prezzo}_{iniziale}-Sconto=&{Prezzo}_{finale}\\
	{Prezzo}_{iniziale}-\dfrac{{Prezzo}_{iniziale}\cdot {Sconto}_{Percentuale} }{100}=&{Prezzo}_{finale}\\
	{Prezzo}_{iniziale}\left(1-\dfrac{{Sconto}_{Percentuale} }{100}\right)=&{Prezzo}_{finale}\\
	{Prezzo}_{iniziale}=&\dfrac{{Prezzo}_{finale}}{1-\dfrac{{Sconto}_{Percentuale} }{100}}
\end{align*}
A questo punto armiamoci di calcolatrice e iniziamo.
\begin{align*}
	{Prezzo}_{finale}=&\mbox{\EUR{23.75}}\\
	{Sconto}_{Percentuale}=&\SI{5}{\percent}\\
{Prezzo}_{iniziale}=&\dfrac{23.75}{1-\dfrac{5}{100}}\\
=&\mbox{\EUR{25}}
\end{align*}
\chapter{Incrementi e decrementi}
Vi sono altri problemi che appaiono come somma o differenza fra due valori
\section{Incrementi}
	\begin{esempiot}{Conosco valore finale e incremento percentuale}{}
Un quadro viene venduto a \EUR{5000}  aumentato del \SI{25}{\percent} del suo precedente valore. Quanto era il suo prezzo iniziale?
\end{esempiot}
\begin{align*}
	{Val}_{Finale}=&{Val}_{Iniziale}+Incremento\\
	Incremento=&\dfrac{{Val}_{Iniziale}\cdot {Inc}_{Percentuale} }{100}\\
	{Val}_{Finale}=&{Val}_{Iniziale}+\dfrac{{Val}_{Iniziale}\cdot {Inc}_{Percentuale} }{100}\\
	{Val}_{Finale}=&{Val}_{Iniziale}\left(1+\dfrac{{Inc}_{Percentuale} }{100}\right)\\
	{Val}_{Iniziale}=&\dfrac{{Val}_{Finale}}{1+\dfrac{{Inc}_{Percentuale} }{100}}
\end{align*}
Possiamo ora risolvere il problema
\begin{align*}
	{Val}_{Finale}=&\mbox{\EUR{5000}}\\
	{Inc}_{Percentuale}=&\SI{25}{\percent}\\
	{Val}_{Iniziale}=&\dfrac{5000}{1+\dfrac{25}{100}}\\
	=&\mbox{\EUR{4000}}\\
\end{align*}
\section{Decrementi}