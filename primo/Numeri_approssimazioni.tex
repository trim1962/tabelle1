\chapter{Approssimazione, arrotondamento e troncamento}
\label{cha:Approssimazione}
\section{Approssimazioni per difetto e per eccesso}
Un'approssimazione è la rappresentazione non precisa di una quantità. Un'approssimazione  può essere o per eccesso\index{Approssimazione!per eccesso}\index{Approssimazione!per difetto} o per difetto. Indicando con  $q$ è la quantità e $a$ e l'approssimazione avremo che:
se\[a<q\] l'approssimazione è per difetto altrimenti se \[a>q\] per eccesso.
  \begin{esempiot}{Approssimazioni}{}
	Consideriamo il numero\[q=\num{5,3278432}\]
\end{esempiot}
\begin{align*}
	a=&\num{5}\\
	a=&\num{5,3}\\
	a=&\num{5,32}\\
	a=&\num{5,327}\\
\end{align*}
Sono tutte approssimazioni per difetto di q.
\begin{align*}
	a=&\num{6}\\
	a=&\num{5,4}\\
	a=&\num{5,34}\\
	a=&\num{5,328}\\
\end{align*}
Sono tutte approssimazioni per eccesso di q.
\section{Troncamento}\index{Troncamento}
Un troncamento di un numero decimale è riscrivere un numero eliminando le cifre di un numero decimale da una posizione scelta in poi. 
  \begin{esempiot}{Troncamento}{}
	Consideriamo il numero\[q=\num{5.689547155}\]
	il suo troncamento dalla terza cifra decimale è:
	\[q=\num{5.689}\]
\end{esempiot}
  \begin{esempiot}{Troncamento}{}
	Consideriamo il numero\[q=\num{8.45698202124}\]
	il suo troncamento dalla quinta cifra decimale è:
	\[q=\num{8.45698}\]
\end{esempiot}
\section{Arrotondamenti}\index{Arrotondamenti}
Come si decide se approssimare per difetto o per eccesso? Si utilizza il metodo di arrotondamento. Utilizziamo la seguente regola se vogliamo arrotondare un numero a una certa posizione decimale se la successiva cifra è \num{0}, \num{1}, \num{2}, \num{3}, \num{4} si tronca alla posizione. Se la successiva cifra è \num{5}, \num{6}, \num{7}, \num{8}, \num{9}. Si aumenta di uno.
  \begin{esempiot}{Arrotondamento}{}
	Consideriamo il numero\[q=\num{8.45698202124}\]
	arrotondando dalla terza cifra decimale è:
	\[q=\num{8.457}\] visto che la quarta cifra decimale è \num{9}
\end{esempiot}
  \begin{esempiot}{Arrotondamento}{}
	Consideriamo il numero\[q=\num{8.45698202124}\]
	arrotondando dalla terza cifra decimale è:
	\[q=\num{8.45698}\] visto che la quarta cifra decimale è \num{2}
\end{esempiot}
